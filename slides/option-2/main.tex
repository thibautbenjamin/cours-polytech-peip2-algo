\documentclass[aspectratio=169]{beamer}

\usepackage[utf8]{inputenc}
\usepackage[french]{babel}
\usepackage[T1]{fontenc}
\usepackage{tikz}
\usepackage{tikz-cd}
\usepackage{listings}
\usepackage{lstautogobble}
\usepackage{tcolorbox}
\tcbuselibrary{listings}
\tcbuselibrary{listingsutf8}
\usetikzlibrary {graphs,graphdrawing}
\usegdlibrary{trees}
\usepackage{pgfplots}

\usetheme[sectionpage=progressbar, subsectionpage=simple]{metropolis}

\newtheorem{exercice}{Exercice}

\usepackage{macros}

\setbeamertemplate{itemize item}{{\usebeamercolor[fg]{alerted
      text}{${\scriptstyle \blacktriangleright}$}}}


\usepackage{xpatch}
\makeatletter
\newlength{\my@beamer@itemsepi}\setlength{\my@beamer@itemsepi}{3ex}
\newlength{\my@beamer@itemsepii}\setlength{\my@beamer@itemsepii}{1.5ex}
\newlength{\my@beamer@itemsepiii}\setlength{\my@beamer@itemsepiii}{1.5ex}
\newcommand{\my@beamer@setsep}{%
  \ifnum\@itemdepth=1\relax
    \setlength\itemsep{\my@beamer@itemsepi}
  \else
    \ifnum\@itemdepth=2\relax
      \setlength\itemsep{\my@beamer@itemsepii}
    \else
      \ifnum\@itemdepth=3\relax
        \setlength\itemsep{\my@beamer@itemsepiii}
      \fi\fi\fi}
\xpatchcmd{\itemize}
{\def\makelabel}{\my@beamer@setsep\def\makelabel}{}{}
\xpatchcmd{\beamer@enum@}
{\def\makelabel}{\my@beamer@setsep\def\makelabel}{}{}
\newcommand\setlistsep[3]{%
  \setlength{\my@beamer@itemsepi}{#1}%
  \setlength{\my@beamer@itemsepii}{#2}%
  \setlength{\my@beamer@itemsepiii}{#3}%
}
\makeatother

\setlistsep{6.5ex}{2ex}{2ex}

%%%
\definecolor{lbcolor}{rgb}{0.1,0.1,0.1}
\definecolor{commentcolor}{rgb}{0.4,0.4,0.4}
\definecolor{keywordcolor}{HTML}{531ab6}
\definecolor{stringcolor}{HTML}{005f5f}

\lstset{
  basicstyle=\small\ttfamily\color{black},
  commentstyle=\rmfamily\color{commentcolor},
  keywordstyle=\bfseries\color{keywordcolor},
  showspaces=false,
  showstringspaces=false,
  stringstyle=\color{stringcolor},
  tabsize=2,
}

\newtcblisting{slidelisting}{
      arc=5mm,
      top=0mm,
      bottom=0mm,
      left=0mm,
      right=0mm,
      boxrule=1pt,
      listing only,
      listing options={language=C++},
      width=\textwidth
    }
    \NewTCBInputListing{\slideinputlisting}{ O{0} O{9999999} m }{
      listing file = #3,
      size = small,
      arc=5mm,
      top=0mm,
      bottom=0mm,
      left=0mm,
      right=0mm,
      boxrule=1pt,
      listing only,
      listing options={language=C++, firstline=#1, lastline=#2},
      width=\textwidth
}


\NewDocumentCommand{\codeslide}{ O{4} O{9999999} m}{
  \slideinputlisting[#1][#2]{#3}
  \onslide<2>
  \slideinputlisting[0]{#3.res}}


\title{Introduction à la vérification de programmes}
\subtitle{Polytech Paris-Saclay, PEIP 2, Informatique Option S3}
\author{Thibaut Benjamin, Henri Saudubray, Philippe Volte--Vieira}
\date{18 Décembre 2025}
\institute{Cours 2}
\begin{document}

\maketitle


\begin{frame}
  \frametitle{Format et date de l'Examen}
  \begin{itemize}
  \item Examen de 1h30 (2h pour les tiers-temps)
  \item Le 9 Janvier 2026
  \item Au format TP, avec des exercices type
  \end{itemize}
\end{frame}

\begin{frame}
  \frametitle{Récapitulatif}

  \begin{itemize}
  \item La semaine dernière, nous avons vu comment utiliser la méthode du
    variant pour prouver la terminaison des boucles.
  \item \textbf{\alert{Rappel}}Un variant est une quantité entière positive
    associée à une boucle, et qui décroît strictement à chaque fois que l'on
    entre dans cette boucle.
  \end{itemize}

\end{frame}

\begin{frame}
  \frametitle{Séance du jour}

  \begin{itemize}
  \item Les contrats de fonctions
    \begin{itemize}
    \item Pré-conditions: ce que l'on doit vérifier à chaque appel de la
      fonction.
    \item Les variants pour les fonctions récursives.
    \item Post-conditions: ce que la fonction garantit.
    \end{itemize}


  \item Contrats pour les fonctions de tri
  \end{itemize}
\end{frame}

\section{Les contrats de fonction}

\subsection{Les pré-conditions}

\begin{frame}[fragile]
  \frametitle{Un premier exemple}

  \slideinputlisting{code/exemple1.cpp}

  \href{https://www.why3.org/try/?name=test.c&lang=micro-C&code=A7Ou5renvoie0le5dernier5element0du5tableauN1intyrenvoieDernier%2F7nrzn7rr1tab7WYI7H4returnr72nA7sAz17YwN79N}{\alert{Démo
      en Why3}}

  \begin{itemize}
  \item<2-> A quoi correspond l'obligation de preuve générée par Why3?
  \item<2-> Quelle information manque pour satisfaire la condition?
  \end{itemize}

\end{frame}

\begin{frame}[fragile]
  \frametitle{Pré-conditions d'une fonction}
  On peut attacher une précondition à une fonction (mot-clé
  \lstinline|@requires| en Why3)

  \slideinputlisting{code/exemple2.cpp}
  \begin{itemize}
  \item A l'intérieur de la fonction on suppose que les préconditions sont
    vraies


    \href{https://www.why3.org/try/?name=isqrt.c&lang=micro-C&code=A7Ou5renvoie0le5dernier5element0du5tableauN1intyrenvoieDernier%2F7nrzn7rr1tab7WYI7J7OO16requiresz17Ryls4lengthjl7IwF4returnr72mA7sAk7YwN79N}{\alert{Démo}}

    \vspace{-1cm}

  \item A chaque appel de la fonction, il faut prouver les préconditions

    \href{https://www.why3.org/try/?name=isqrt.c&lang=micro-C&code=A7Ou5renvoie0le5dernier5element0du5tableauN1intyrenvoieDernier%2F7nrzn7rr1tab7WYI7J7OO16requiresz17Ryls4lengthjl7IwF4returnr72mA7sAk7YwN79NNe2main7HI7JrkBkz5mHqqz0747y0107wHnAngnnz9nHnnz2nnz8nHnnz3nnz7nHnnz4nnz6nHcuDpuGASh7rhRHRYiNT}{\alert{Démo}}
  \end{itemize}
\end{frame}


\begin{frame}[fragile]
  \frametitle{Pré-conditions pour prouver la correction}

  \begin{itemize}
  \item Dans l'exemple précédent, on a donné une précondition pour vérifier
    l'absence de bugs: pas d'accès à des zones du tableau non définies.
  \item Si on veut garantir que la fonction renvoie effectivement le dernier
    élément du tableau, il faut une précondition plus précise.
  \end{itemize}

  \href{https://www.why3.org/try/?name=isqrt.c&lang=micro-C&code=A7Ou5renvoie0le5dernier5element0du5tableauN1intyrenvoieDernier%2F7nrzn7rr1tab7WYI7J7OO16requiresn7Sy4lengthkm7IwHb4Quelle3autre0pr%2FD%2Bpycondition%2F5ajouter70F4returnk72fA7sAz17YwN79NNX2main7HI7JrkBkz5mHqqz0747y0107wHnAngnnz9nHnnz2nnz8nHnnz3nnz7nHnnz4nnz6nHcuCpuFA7nZ7rAhuKHHuK0Si0on3donne0la6mauvaise6longueurko0neBHl2peut1paszv%2FD%2Bp4rifierjuM%2FD%2BpuMHmXydernierFaux%2FXXAXPdBXXHuNz0SN79}{\alert{Démo}}
\end{frame}

\begin{frame}[fragile]
  \frametitle{Exercice: affichage d'un tableau}

  Donner les préconditions pour la fonction suivante
  \slideinputlisting[3]{code/exercice1.cpp}

  \href{https://www.why3.org/try/?name=isqrt.c&lang=micro-C&code=A7iA5includeA7xA3stdio7tAzhA7zNN2voidyafficherTableau%2FA7n1intzn7rr1tabB7WYI7J7OuA0Pr%2FD%2Bpycondition%2FB70H1forjlzi7yz07wqAYAeqqA7KKI7L7OO15varianto7sonH4printfA7HBkAqB7BrY72q7YIwD79Nt}{\alert{Démo}}

  \alert{\textbf{Attention}}: Même avec la bonne précondition, why3 ne saura pas
  nécessairement montrer l'absence de bugs!

\end{frame}


\begin{frame}[fragile]
  \frametitle{Exercice: Addition bête}

  Donner le variant et les préconditions pour la fonction suivante
  \slideinputlisting[3]{code/exercice2.cpp}

  \href{https://www.why3.org/try/?name=isqrt.c&lang=micro-C&code=A7iA5includeA7xA3stdio7tAzhA7zNN1intyadditionBete%2F7nrzx7rrzy7I7J7OuAyPrecondition%2F70H3whilekniz0lLlA5VariantBlBHh7ys7qz17wHios7sppD79H4returnlqNq}{\alert{Démo}}

\end{frame}

\begin{frame}[fragile]
  \frametitle{Exercice: Afficher de k en k}

  Donner le variant et les préconditions pour la fonction suivante
  \slideinputlisting[3]{code/exercice3.cpp}

  \href{https://www.why3.org/try/?name=isqrt.c&lang=micro-C&code=A7iA5includeA7xA3stdio7tAzhA7zNN2voidyafficherEnK%2F7n1intzn7rr1tab7WYrrzk7I7H7OuAyPrecondition%2F70H1forimzi7yz07wqXdqqos7qffLfA5VariantfH4printfA7HBkAlB7BrZ72q7YIwD79NtN}{\alert{Démo}}

  \onslide<2->{\alert{\textbf{Remarque}}: Dans cet exemple, c'est la
    précondition qui rend le variant correct. Comment interpréter cela?}

\end{frame}

\begin{frame}[fragile]
  \frametitle{Exercice: pgcd}

  Donner les préconditions pour la fonction suivante
  \slideinputlisting[3]{code/exercice4.cpp}

\href{https://www.why3.org/try/?name=isqrt.c&lang=micro-C&code=A7iA5includeA7xA3stdio7tAzhA7zNN1int2pgcd7nrza7rrzb7I7H7OuAyPreconditions%2FCBH3whileAllB7AymmJ7OO15varianto7qo7wH0ifkodo7o77p7ys7sok79H2elseAmpmsnmmmFtH4returnqqNqN}{\alert{Démo}}

\end{frame}


\subsection{Les variants de fonctions récursives}
\begin{frame}
  \frametitle{Principe}
  \begin{itemize}
  \item Pour une fonction récursive, on peut ajouter un variant au contrat de la
    fonction.
  \item Dans ce contexte, un variant est un entier positif qui décroit à chaque
    appel de la fonction
  \end{itemize}
\end{frame}

\begin{frame}[fragile]
  \frametitle{Exemple}

  \slideinputlisting{code/exemple6.cpp}

  Combiner la précondition et le variant permet de prouver l'arrêt du programme.

  \href{https://www.why3.org/try/?name=isqrt.c&lang=micro-C&code=A7iA5includeA7xA3stdio7tAzhA7zNN1intyfactorielle%2F7nrzn7I7J7OO16requiresqA7TyAz07wHo5variantoqH0ifjq7Sym7o774returnz1k79Hql7pbiqA7sAl7IwNl}{\alert{Démo}}
\end{frame}

\begin{frame}[fragile]
  \frametitle{Exercice: Fibonacci}

  Donner les préconditions et le variant pour prouver l'arrêt de la fonction de
  calcul des nombres de Fibonacci.

  \slideinputlisting{code/exercice7.cpp}

  \href{https://www.why3.org/try/?name=isqrt.c&lang=micro-C&code=A7iA5includeA7xA3stdio7tAzhA7zNN1int2fibo7nrzn7I7J7OuA5ContratH0ifno7Ryz1m4returnp7w79HqhAloA7sAl7o7qnnnAnAz27IwNj}{\alert{Démo}}

\end{frame}

\begin{frame}[fragile]
  \frametitle{Exercice: PGCD récursif}

  \'Ecrire les préconditions et le variant pour prouver la terminaison de
  l'algorithme du pgcd récursif.

  \slideinputlisting{code/exercice8.cpp}
  \href{https://www.why3.org/try/?name=isqrt.c&lang=micro-C&code=A7iA5includeA7xA3stdio7tAzhA7zNN1int2pgcd7nrza7rrzb7I7J7OuA5ContratH0ifll7Symm4returnp7w79H2elsekkoekkkeAnpemA7sAq7IwhHh77jjjkjkAjAqjjNtN}{\alert{Démo}}

\end{frame}

\subsection{Les post-conditions}

\begin{frame}[fragile]
  \frametitle{Une question pour le moment insoluble}

  \slideinputlisting[3]{code/exemple3.cpp}

\end{frame}

\begin{frame}[fragile]
  \frametitle{Solution: une post-condition}
  \begin{itemize}
  \item Pour pouvoir prouver la précondition, il faut un moyen de dire que la
    fonction \lstinline|abs| renvoie tojours un entier positif
    \vspace{-1cm}
  \item C'est le rôle d'une post condition (mot-clé \lstinline|@ensures| en Why3)

    \slideinputlisting{code/exemple4.cpp}

    \href{https://www.why3.org/try/?name=isqrt.c&lang=micro-C&code=A7Ou4valeur5absolueN1int1abs7nrzx7I7H7OO15ensuresC4result7Tyz07wH0ifjkoo7o774returnol79H2elseoo7sAnnnNtN}{\alert{Démo}}
  \end{itemize}
\end{frame}

\begin{frame}[fragile]
  \frametitle{Exercice: Addition bête}

  Donner les post-conditions pour la fonction suivante
  \slideinputlisting[3]{code/exercice2.cpp}

  \href{https://www.why3.org/try/?name=isqrt.c&lang=micro-C&code=A7iA5includeA7xA3stdio7tAzhA7zNN1intyadditionBete%2F7nrzx7rrzy7I7J7OuAyPrecondition%2F70H3whilekniz0lLlA5VariantBlBHh7ys7qz17wHios7sppD79H4returnlqNq}{\alert{Démo}}
\end{frame}

\begin{frame}[fragile]
  \frametitle{Exercice: PGCD}

  \slideinputlisting[3]{code/exercice5.cpp}

  \href{https://www.why3.org/try/?name=isqrt.c&lang=micro-C&code=A7iA5includeA7xA3stdio7tAzhA7zNN1int2pgcd7nrza7rrzb7I7H7OuAyPreconditions%2FCBH3whileAllB7AymmJ7OO15varianto7qo7wH0ifkodo7o77p7ys7sok79H2elseAmpmsnmmmFtH4returnqqNqN}{\alert{Démo}}

  \textbf{\alert{Indice}}: Commencer par donner les conditions disant que c'est
  un diviseur commun, puis la condition disant que c'est le plus grand.

  Pour la dernière post-condition, il faudra utiliser le mot-clé
  \lstinline|forall|.
\end{frame}

\begin{frame}[fragile]
  \frametitle{Définition de fonctions et prédicats}

  \begin{itemize}
  \item Pour s'aider à donner les conditions, on peut définir des fonctions et
    prédicats auxilliaires.

    \slideinputlisting[1][2]{code/exemple5.cpp}

    \vspace{-1cm}

  \item On peut même utiliser des quantificateurs (\lstinline|forall|,
    \lstinline|exists|)

    \slideinputlisting[4][7]{code/exemple5.cpp}
  \end{itemize}

\end{frame}


\section{Contrats pour le tri par insertion récursif}

\begin{frame}[fragile]
  \frametitle{Rappel: Le tri par insertion}

  \begin{itemize}
  \item Pour procéder au tri par insertion, on commence par trier le premier
    élément du tableau, puis les deux premiers, puis les trois premiers, etc
  \item A l'étape, on performe une insertion, permettant d'insérer
    l'élément à la position \(n\) dans la bonne position parmi les \(n\)
    premiers éléments déjà triés du tableau.
  \end{itemize}

\end{frame}


\begin{frame}[fragile]
  \frametitle{Exercice: Prédicat de tri}

  \begin{itemize}
  \item Définir un prédicat \lstinline|trieJusque| prenant un argument un tableau
    d'entier et un entier \lstinline|n| et qui détermine si les \(n\) premières
    entrées du tableau sont triées.


  \item Définir un prédicat \lstinline|trie| prenant un argument un tableau
    d'entier et qui détermine si le tableau est trié ou non.
  \end{itemize}

\end{frame}

\begin{frame}[fragile]
  \frametitle{Exercice: La fonction d'insertion}

  Donner le contrat associé à la fonction d'insertion dans le tableau.

  \slideinputlisting{code/exercice9.cpp}

  \textbf{\alert{Remarque}}: Le code de la fonction n'est pas nécessaire pour
  donner le contrat: modularité.

  \href{https://www.why3.org/try/?name=isqrt.c&lang=micro-C&code=A2voidyinsertion%2F7n1int1tab7WYrrzn7I7J7Ou5ContratHs2CODE0DE0LA6FONCTIONN79}{\alert{Démo}}

\end{frame}

\begin{frame}[fragile]
  \frametitle{Exercice: La fonction de tri par insertion récursive}

  Donner le contrat associé à la fonction récursive de tri par insertion

  \slideinputlisting{code/exercice10.cpp}

  \href{https://www.why3.org/try/?name=isqrt.c&lang=micro-C&code=A2voidytriInsertionRec%2F7n1int1tab7WYrrzn7I7J7OuA5ContratH0iflo7Syz07o7b9H2elseA77Lele7rjA7sAz17IwHyinsertion%2FAmmmmoD79Nt}{\alert{Démo}}

\end{frame}

\begin{frame}[fragile]
  \frametitle{Exercice: La fonction de tri par insertion récursive}

  Donner le contrat associé à la fonction récursive de tri par insertion

  \slideinputlisting{code/exercice11.cpp}

  \href{https://www.why3.org/try/?name=isqrt.c&lang=micro-C&code=A2voidytriInsertion%2FA7n1int1tab7WYrrzn7I7J7OuA5ContratHytriInsertionRec%2FAll7rm7IwN79}{\alert{Démo}}

\end{frame}

\begin{frame}[fragile]
  \frametitle{Qu'a-t-on prouvé}
  \begin{itemize}
  \item La logique de la récursivité est correcte.
  \item Des lors que l'on fournit une fonction d'insertion correcte, notre tri
    par insertion est correct.
  \end{itemize}
\end{frame}

\begin{frame}[fragile]
  \frametitle{Une spécification incomplète?}
  \begin{itemize}
  \item Trouver une fonction qui satisfait le contrat précédent, mais qui n'est
    pas le tri d'un tableau.

  \item Comment affiner notre contrat pour complètement spécifier le tri?

    \alert{Ne pas le faire avec Why3.}
  \end{itemize}
\end{frame}




\end{document}

%%% Local Variables:
%%% mode: LaTeX
%%% TeX-master: t
%%% TeX-engine: luatex
%%% End:
